\documentclass[12pt]{beamer}
\usepackage[utf8]{inputenc}
\usepackage[portuguese]{babel}
\usepackage{graphicx}
\graphicspath{{./images}}
\usepackage{colortbl}
\usepackage{color}
\usepackage{cite}
\usepackage{breqn}

\definecolor{azul}{rgb}{0,0,.5}
\setbeamertemplate{navigation symbols}{}

\usetheme{Frankfurt}
\usecolortheme[named=azul]{structure}

%% Definindo o Autor e o título
\newcommand{\prof}{Claudio R. M. Mauricio}
\newcommand{\materia}{Processamento de Imagens Digitais}

\author[Aluno:~Victor E. Almeida]{Victor Emanuel Almeida}

\title{Apresentação de \materia}
\subtitle{Demonstrando o software}
\date{\today}
\institute{UNIOESTE}

\begin{document}
\frame{\titlepage}

\begin{frame}
\frametitle{Conteúdo}
\tableofcontents
\end{frame}

\section{Introdução}\label{Introdução}
\begin{frame}
    \frametitle{Softwares utilizados}
    \begin{itemize}
    \item\textbf{Linguagem de programação}: C++;
    \item\textbf{Framework para interfaces gráficas}: wxWidgets\cite{docsWx};
    \item\textbf{Framework para visão computacional}: opencv;
    \end{itemize}
\end{frame}

\begin{frame}
    \frametitle{Estrutura do projeto}
\end{frame}

\begin{frame}
    \frametitle{Agradecimentos}
    \centering
    \Huge{Obrigado pela atenção}
\end{frame}

\section{Referências}\label{Referências}
\begin{frame}[allowframebreaks]
    \frametitle{Referências} 
    \bibliography{ref}
    \bibliographystyle{abbrv} % funciona
\end{frame}

\end{document}
